\documentclass{diploma}
\usepackage{cmap} % for serchable pdf's
\usepackage[T2A]{fontenc} 
\usepackage[utf8]{inputenc}
\usepackage[english,russian]{babel}
\usepackage{indentfirst}
\usepackage[colorlinks=true,urlcolor=black,linkcolor=black,filecolor=black,citecolor=black,unicode,pdftex]{hyperref}
\usepackage{graphicx}
\usepackage{caption}
\usepackage{pdfpages}
\usepackage{amsmath}
\usepackage{dot2texi}
\usepackage{tikz}
\usetikzlibrary{shapes,arrows}
\usepackage{listings}
%\usepackage[titletoc]{appendix} 
%\usepackage{citehack}
\renewcommand{\baselinestretch}{1.5}


% to economize paper (for printing) uncomment next 5 lines
%\textwidth=190mm
%\textheight=250mm
%\topmargin=-20mm
%\oddsidemargin=-15mm
%\evensidemargin=-15mm



\begin{document}
\sloppy

\title{Генератор синтаксических анализаторов для неоднозначных контекстно-свободных грамматик}
\author{Григорьева Семёна Вячеславовича}
\university{Санкт-Петербургский Государственный Университет}
\facility{Математико-механический факультет}
\group{461}
\position{студента}
\chair{Кафедра системного программирования}
\leaderPosition{к.ф.-м.н.}
\leader{А.С. Лукичев}
\criticPosition{д.ф.-м.н., проф.}
\critic{Б.К. Мартыненко}
\chairLeaderPosition{д.ф.-м.н., проф.}
\chairLeader{А.Н. Терехов }
\city{Санкт-Петербург}
\yr{2010}

\maketitle
\includepdf{SemenDiplomaTitle_en.pdf}
\setcounter{page}{2}
\tableofcontents
\newpage


\section{��������.}	

������ �������������������  �������������  �������� ��������� ������ ���������� � ����������� �������������� ������������.

��� ����������� ����� ������ (� �������� � ����������) ������ ����������� ����������-��������� ����������. ���������� ����������� ������������� ��� ������������, ������� ���������� � ������������, ����� �������� ����� ����������, �� ��� ���� ��� ��������� ���� �������������� ~\cite{CurrentParsTechn}. ������� ���������� ���� ������ �������� � ������� ������������� ����������-��������� ����������.

��� ��������� ���������� ��������� ����� ���������� ����������� ����� ������������� ����������. ������,��������, ��������� ������ ������� �������� � ��������� �������� ���������� � ���������� ~\cite{CurrentParsTechn}, ������� ���������� ��������� �������.
��� ������� �������� ���������� �������. 

��� ������� ������� ���� ����� ������������ ������������ GLR ���������� � ��������������� ����������� ���������� ������������\cite{CurrentParsTechn}. �������������, GLR-�������� ��������� ��������������� � ���������� �� ������ ���������. �� �����  ������� ������������ ���������� ���������� �����, ������� ������ �������. ������������ ��� ���������� � ��������������. 
      
������� ������������ GLR-��������� �������� ��������� ������������� ���������. \linebreak ����������, ����������� � ������� ������� ���������, � ���������� ������� ������ �� ������������ ������, � ��������� �������� - ���, ������� ����� ���������, ��������� ����������� �������, � ����� ,��� ������� � ����� ������������ ���������� ���������, ������� ���� ��� ��� ����������� ������ ������� ������/��������.   

����� ��������, ��� �� ������������������ ����� ����������, ������� ��������� "�����������" ��� LR-������������, ������������� ��� ��������. �� ����������� ���� � ����������� ������������������/����� ����������� ������ GLR-�������� �������� �������� ���������������.

����������� ������� ���������� ����������� ����������-��������� ���������. �������� ���������� ������ ����� ��������� ���� ������ ���������� �����, ������ � ��������. 


\clearpage
\section{���������� ������}

������ ������ ������ ��������:
\begin{itemize}
	\item ������� ����������-����������� ��������;
	\item ������� ����������� ����������-����������� ��������� ��� ������ � EBNF ������������;
	\item ������� ����������� ������ ���������� ��������� ��� ������ � ������������ ������������� ����������-���������� ������������;
	\item ����������� �������� ���������� �������������� ������������, ���������� �� ����������-���������� ��������� � ������������ ����������� ������� ��������� �����:
		\begin{itemize}
			\item ������ � ������������ �� ������������
			\item ������ s-����������� ������������
		\end{itemize}
\end{itemize}

� ���������� ������ ������ ���� ������ �������� �����������, ��� �������� ���������� ������� ��� ����������.
\begin{itemize}
	\item ����������� ������ � �������������� ������������. ��� ����� ���������� ���������, ��� ��� ������������� ���������� ���������� ���������� ��� ��������� �������� ������ ������� ������.
	\item ����������� ������������ �������� ���� ���������, ������������� � ���������� ��������� ������. %��� ������� ���������� � EBNF. �������� ����������� ���������� ���������, ��� ������� ���������� �������� �������� - ������������������, ������������, ���������. ����� �������� �������� �� ������� �������������� ������� ���������� � �������� ������. ��� ����� ����� ��������� ������������ ������ ������ ������� ����������. � ��� �� ������ ���� ����� ���������� � ������������.
	\item ��������� s-���������� ���������. %���������� �������� ������������ ���������� ��������� � ������ ������������� ����������. ��� ����� ����������, ����� ���� �������� ��� ��������� ���������� ���������� � ����� �������� � ��������� ��������� �������� ��������� (��������, ���������, ��� ��� ������� � ��������� �������� ������ �� �����, �� �����  �� ��������� ������ ����������). ��� �� ���������� �������� ����������� ���������� ��������� � ������ ����������� �� ����������.
\end{itemize}

����� ����, ���������� ������� �������� ������� ������������ �����������.% ����� ��������� ���������� �������� ����������� �������������� �������������� ������ � ������� ���������� ���������.
\clearpage
\section{Основные определения}

Определим ряд понятий и введём некоторые обозначения, необходимых для дальнейшего изложения.
\\
\\
	{\bfseries Определение 1.} \textit{Конструкции регулярнх выражений}: будем говорить, что грамматика (правило) содержит конструкции регулярных выражений, если в записи правых частей правил используются элементы синтаксиса регулярных выражений (альтернатива, замыкание).
\\
	{\bfseries Определение 2.} \textit{Раскрытие конструкций регулярных выражений}: будем называть раскрытием конструкций регулярных выражений такое преобразование грамматики (правила), при котором исходная грамматика заменяется на эквивалентную, не содержащую конструкций регулярных выражений.
\\
	{\bfseries Определение 3.} \textit{Дерево вывода строки в EBNF-грамматике}: упорядоченное помеченное дерево $D$ называется деревом вывода в EBNF-грамматике $G(S)=(N,T,P,S)$, если выполнены следующие условия:

\begin{enumerate}
	\item корень дерева $D$ помечен $S$;
	\item каждый лист помечен либо $a \in T$, либо $\varepsilon$;
	\item каждая внутренняя вершина помечена нетерминалом;
	\item если $N$ -- нетерминал, которым помечена внутренняя вершина и $X_1,...,X_n$ - метки ее прямых потомков в указанном порядке, то существует правило $N \rightarrow Y_1...Y_n \in P$ такое, что строка $X_1...X_n$ пораждается регулярным выражением $Y_1...Y_n$.
\end{enumerate}
	{\bfseries Определение 4.} \textit{Непосредственная поддержка EBNF-грамматик}: будем говорить, что инструмент непосредственно поддерживает EBNF-грамматики, если он работает с ними без раскрытия конструкций регулярных выражений.
\\
 	{\bfseries Определение 5.} \textit{Побочный эффект}: будем говорить, что функция или атрибут обладают побочным эффектом, если в процессе их вычислений возможно читать и модифицировать значения глобальных переменных, осуществлять операции ввода/вывода, реагировать на исключительные ситуации, вызывать их обработчики.



Для примеров псевдокода, приводимых далее, будем использовать синтаксические соглашения, принятые в языке программирования F\#.

%\begin{itemize}

	%\item GLR: Generalized LR Parsing~\cite{CurrentParsTechn}

	%%\item ДКА: детерминированный конечный автомат. Такой автомат, в котором для каждой последовательности входных символов существует лишь одно состояние, в которое автомат может перейти из текущего~\cite{DrgBook}.
	
	%%\item НКА: недетерминированный конечный автомат~\cite{DrgBook}.
	
	%\item Замыкание: q* = q$ \bigcup \{B\rightarrow.c | A \rightarrow a.Bb \in $q*$\} \bigcup \{x\stackrel{}{\rightarrow}.x | A\stackrel{}{\rightarrow} a.xb \in $q*$\}$~\cite{DrgBook}
	
	%%\item LR-ситуация: продукция с точкой в некоторой позиции правой части~\cite{DrgBook}.

%\end{itemize}

\clearpage
\section{Обзор}


\subsection{Алгоритм анализа}

Одно из основных требований к инструментам автоматической генерации анализаторов, применяемым при решении задач реинжиниринга -- возможность работать с неоднозначными грамматиками, так как с их помощью часто описываются устаревшие языки.

Существует несколько подходов к реализации алгоритма, позволяющего работать с грамматиками, содержащими неоднозначности:
\begin{itemize}
  \item {\bfseries алгоритм Томиты} (GLR-алгоритм)~\cite{Practical Guide}, основанный на организованном в виде графа стеке;
  \item {\bfseries алгоритм Эрли} (Early)~\cite{Practical Guide}, основа которого -- специальным образом определённое состояние;
  \item {\bfseries рекурсивно-восходящий алгоритм} (recursive-ascent)~\cite{RECURSIVE-ASCENT PARSING}~\cite{RecursiveAscentParsing}, основанный на наборе взаимно-рекурсивных функций, эмулирующих переходы между состояниями, и механизме запоминания результатов предыдущих вычислений;
  \item {\bfseries CYK}~\cite{Practical Guide}
  \item {\bfseries Unger}~\cite{Practical Guide}
\end{itemize}
       
В настоящее время наиболее популярным в практическом применении является алгоритм Томиты. Существует ряд инструментов, основанных на этом алгоритме:
\begin{itemize}
	\item
	 ASF+SDF~\cite{ASF+SDF} (Algebraic Specification Formalism + Syntax Definition Formalism) -- генератор с широкими возможностями, но достаточно сложным входным языком. Является SGLR-инструментом (Scannerless, Generalized-LR).
	
	\item
	 Bison~\cite{Bison} -- развитие инструмента YACC. Все грамматики, созданные 	для оригинального YACC, будут работать и в Bison. Является одним 	из самых популярных и совершенных "потомков" \ YACC. При включении 	соответствующей опции использует GLR-алгоритм (по умолчанию LALR).
	
	\item
	Elkhound~\cite{Elkhound} -- позиционируется как быстрый и удобный GLR-инструмент, созданный в университете Беркли (США), тем не менее обладает достаточно 	"бедным" \ входным языком (например, он не поддерживает конструкций 	расширенной формы Бэкуса-Наура).

  \item 
  DMS~\cite{DMS} -- инструментарий "DMS Software Reengineering Toolkit" \ включает в себя парсер генератор, основанный на GLR алгоритме.

	\item
  Happy~\cite{Happy} -- парсер генератор с целевым языком Haskell~\cite{Haskell}. Формат описания входной грамматики очень похож на формат классического YACC.

	\item
  Dypgen~\cite{Dypgen} -- GLR-инструмент, обладающий такими особенностями как возможность удалять и добавлять правила во время синтаксического анализа, специфический способ задания приоритетов операций.

\end{itemize}

Все вышеперечисленные инструменты реализуют стековый механизм анализа. Так же существует ряд других инструментов, основанных на том же алгоритме: 
APaGeD~\cite{APaGeD}, 
DParser~\cite{DParser}, 
eu.h8me.Parsing~\cite{h8me}, 
GDK~\cite{GDK}, 
SmaCC~\cite{SmaCC}, 
Tom~\cite{Tom}, 
UltraGram~\cite{UltraGram}, 
Wormhole~\cite{Wormhole}. Все они реализуют GLR или SGLR алгоритм и ни один из них не реализует непосредственной поддержки EBNF-грамматик.

Интересующий нас рекурсивно-восходящий алгоритм реализуется на текущий момент только одним инструментом: Jade~\cite{Jade}. Jade -- это генератор рекурсивно-восходящих LALR(1) парсеров с целевым языком С. При реализации данного инструмента возникла серьёзная проблема появилась, связанная с большим объёмом кода целевого парсера. Так как при построении детерминированного парсера необходимо генерировать процедуры для каждого состояния, то объём кода быстро растёт с ростом количества правил в грамматике. Так, например, для языка Java объём кода составляет примерно 4 мегабайта~\cite{Jade}. В Jade  эта проблема решается путём создания глобальной структуры(массива состояний), где хранится информация, позволяющая переиспользовать процедуры.

Однако существует подход к реализации рекурсивно-восходящего алгоритма, позволяющий решить проблему объёма кода~\cite{Non-det-rec-asc}. С использованием этого подхода можно получить алгоритм для недетерминированного анализа, основанный всего на двух взаимно-рекурсивных функциях и позволяющий реализовать непосредственную поддержку EBNF-грамматик.

Таким образом, выяснено, что на текущий момент нет реализаций недетерминированного рекурсивно-восходящего алгоритма, реализующего непосредственную поддержку EBNF-грамматик. Единственная реализация рекурсивно-восходящего алгоритма имеет проблемы с объёмом целевого инструмента, но существует решение этой проблемы.


\subsection{Атрибутные грамматики. Подходы к вычислению атрибутов}

При работе с неоднозначными грамматиками выдвигаются особые требования к алгоритму вычисления атрибутов. Это связано с тем, что в качестве атрибута пользователь может указать действие, обладающее побочным эффектом (например, печать на экран). При наличии таких атрибутов нельзя проводить вычисления непосредственно в процессе анализа, так как в момент разбора не возможно определить, завершиться ли текущая ветвь удачно. В ситуациях, когда при непосредственном вычислении ветвь завершилась неудачно, могут быть совершены лишние действия (например, лишняя печать на экран).

Были рассмотрены два подхода к решению этой проблемы: 
\begin{itemize}

	\item {\bfseries Отложенные вычисления} (continuation passing style, CPS). Непосредственно во время разбора атрибуты не вычисляются. Вычисления откладываются. Строится функция, которая вычисляется только один раз, после удачного завершения разбора.
	
	\item {\bfseries Интерпретация леса вывода} - построение леса вывода и последующее вычисление атрибутов над ним. Первым шагом строится лес вывода, который содержит только деревья, соответствующие успешным вариантам разбора. Следующим шагом над полученным лесом производятся вычисления, соответствующие заданным атрибутам.
	
\end{itemize}

Оба этих подхода гарантируют, что будут выполнены действия, соответствующие только успешным вариантам разбора. Однако второй подход является более удобным для конечного пользователя, так как позволяет явно получить дерево вывода, что упрощает отладку. Именно он и был выбран для реализации.

\clearpage
\section{����������}



\subsection{EBNF ����������}

\subsection{���������� ���� ������}

\subsection{���������� ���������}

YARD ��������� ���������� �������� ��� ����� ����� ���������, ������� �������� �������������������.

���������� �����

������ ������ �������� ���:

������� ������ ������ ������ ������, ��������� � ����������, �������� ���������� ���������� � ������ ����� �������. �� ����� ��� ���������� ���������


TNFA ������������ ��� ������ $(Q$, $\Sigma$, $L$, $T$, $q_0$, $F)$, ��������� ��:
\begin{itmize}
	\item ��������� ��������� ��������� $Q$ 
	\item ��������� ��������� ������� �������� $\Sigma$ 
	\item ��������� ��������� ����� $L$ 
	\item ������� �������� $T:$ $Q \times (Z \cup{ \epsilon })\rightarrow 2^{Q \times L}$
	\item ���������� ��������� $q_0$
	\item ��������� ��������� ��������� ��������� $F$ 
\end{itmize}

����� ������� �������� �������� ���������� �������. ��� ����������� �������� � ���� �����, ��������� ������������� ����. ����� ���������� ����� ����� 
���� "/" ����� �������, ������������ ��������� ��� ������ ��������. 

������ THFA:

<��������>

����� a -- ����������� ������, l -- �����.

����� ������ ������ ���������� �����, ����� �������� � ����� ����������� ������������� ��� ��� ���� ����������� ����������� ���������. 
��� ����� ���������  ����� ������������ ����:
\begin{itemize}
	\item ��� ����������� ������� ����� �����������
		\begin{itemize}
			\item ����: $EltS, EltE$;
			\item �����������������: $SeqS, SeqE$;
			\item ���������: $ClsS, ClsE$;
			\item ������������: $Alt1S, Alt1E, Alt2S, Alt2E$ - ���� ����� ��� ������ �����;
		\end{itemize}
			\item \omega -- "������" �����;
\end{itemize}

����� ��� ����������� ����� ����� �������� �� ���� ����� � ����������� ��������������, ������� ��������� � ����� ������ � ����� ����� � ��� �� �����������.
����� �������, ��������� ����� $L$ ����� ���������� ���:
$$
	L = \left\{t*i|t \in \left\{SeqS, SeqE, EltS, EltE, ClsS, ClsE, Alt1S, Alt1E, Alt2S, Alt2E, \omega \right}, i \in N \right}
$$

������������� TNFA �� ����������� ��������� ���������� ��������������� �������� ��������. ��� ���������� ��������� ���������� ����������� �����.

����������������� �������� ����� ��������� ��������� �������:



��� ����������� ������������� ��� ����� ���������������� �������. ��� ���� ����� ������ �����������

\subsubsection{���������� ������}




\subsubsection{��������� action-����}



\subsubsection{}


\clearpage
\section{Эксперименты}

Был проведён ряд экспериментов, которые позволили оценить некоторые параметры инструмента.

Во всех приведённых ниже тестах грамматики описываются на языке YARD. Так же предположим, что у нас есть сторонний лексер со следующим набором лексем:
\begin{itemize}
  \item PLUS = '+'
  \item MINUS = '-'
  \item DIV = '/'
  \item MULT = '*'
  \item LEFT = '('
  \item RIGHT = ')'
  \item NUMBER = (0..9)+
\end{itemize}

По этому будем предполагать, что на вход инструменту поступает поток лексем.


\subsection{Работа с однозначными грамматиками} 

Необходимо показать, что по однозначной грамматике строится инструмент имеющий линейную временную сложность.
	<Пример грамматики> <Описание эксперимента>


\subsection{ Возможность работы с неоднозначными грамматиками} 

Для этого необходимо проверить, что при неоднозначной грамматике инструмент возвращает все возможные варианты вывода входной строки. В качестве примера была взята следующая грамматика:

\begin{verbatim}
+s : e;
e : e (PLUS | MINUS | MULT | DIV  ) e 
  | LEFT e RIGHT 
  | NUMBER ;
\end{verbatim}

Эта грамматика описывает арифметические выражения без приоритетов. Очевидно, что данная грамматика содержит неоднозначности. 

Рассмотрим несколько примеров входных цепочек:
\begin{itemize}

  \item \verb|[NUMBER; PLUS; NUMBER]|. Существует единственное дерево вывода для данной цепочки:
    \begin{centering}
      \begin{dot2tex}[dot]
       digraph g
       {
          S [label = "S"]
          e1 [label = "e"]
          e2 [label = "e"]
          plus [label = "PLUS"]
          e3 [label = "e"]
          num1 [label = "NUMBER"]
          num2 [label = "NUMBER"]
          S -> e1
          e1 -> e2
          e1 -> plus
          e1 -> e3
          e2 -> num1
          e3 -> num2
       }
      \end{dot2tex}
    \end{centering}

Инструмент возвращает единственное дерево:
\begin{verbatim}
<NODE name="S">
        <NODE name="e">
            <NODE name="e">
                <LEAF name="NUMBER"/>
            </NODE>
            <LEAF name="PLUS"/>
            <NODE name="e">
                <LEAF name="NUMBER"/>
            </NODE>
        </NODE>
</NODE>
\end{verbatim}

\item \verb|[NUMBER; PLUS; NUMBER; PLUS; NUMBER]|. Для данной цепочки существует два дерева вывода и инструмент возвращает оба:
\begin{verbatim}
<NODE name="s">
    <NODE name="e">
        <NODE name="e">
            <NODE name="e">
                <LEAF name="NUMBER"/>
            </NODE>
            <LEAF name="PLUS"/>
            <NODE name="e">
                <LEAF name="NUMBER"/>
            </NODE>
        </NODE>
        <LEAF name="PLUS"/>
        <NODE name="e">
            <LEAF name="NUMBER"/>
        </NODE>
    </NODE>
</NODE>

<NODE name="s">
    <NODE name="e">
        <NODE name="e">
            <LEAF name="NUMBER"/>
        </NODE>
        <LEAF name="PLUS"/>
        <NODE name="e">
            <NODE name="e">
                <LEAF name="NUMBER"/>
            </NODE>
            <LEAF name="PLUS"/>
            <NODE name="e">
                <LEAF name="NUMBER"/>
            </NODE>
        </NODE>
    </NODE>
</NODE>
\end{verbatim}
	
\end{itemize}
	


\subsection{Возможность работы с EBNF-грамматиками} 

Основные конструкции регулярных выражений, для которых необходимо провести проверку - последовательность, альтернатива, замыкание. Важно обратить внимание на соответствие получаемого дерева вывода ожидаемому результату. Для этого нужно проверить соответствие дерева вывода входной грамматике. В нём не должно быть новых терминалов и нетерминалов.
	


\subsection{Поддержка s-атрибутных грамматик}

 Необходимо показать корректность вычисления атрибутов в случае неоднозначной грамматики. Для этого необходимо, чтобы были получены все возможные результаты вычислений и чтобы операции с побочными эффектами работали корректно (например, проверить, что при наличии в атрибутах действия печати на экран, на экран  не выводится лишней информации). Так же необходимо показать возможность вычисления атрибутов в случае расширенной контекстно-свободной  грамматики. 
\\
<Описание эксперимента>
\\

Так же, в ходе этого эксперимента было выявлено, что предложенное решение, когда явным образом строится дерево вывода и для каждого правила строится своя семантическая функция, оказывается удобным. С одной стороны, это позволяет упростить отладку, потому, что всегда можно проверить правильность построения дерева и в отладчике просто проконтролировать вычисление в конкретном узле (мы знаем при свёртке какого правила появился этот узел и знаем какая функция должна вычисляться). С другой -- прямой доступ к лесу вывода позволяет совершать с ним дополнительные операции.  Дополнительную фильтрацию или, например печать, что оказалось полезным при получении результатов экспериментов (печать XML-представления деревьев). 

\clearpage
\section{����������.}

�� ������ ������ ������ ������������, �� ��� ���� ��������� ����������:
\begin{itemize}
\item ���-���
\item ���-���
\end{itemize}

������, ��������� �������:
\clearpage
\begin{thebibliography}{50}

        \bibitem {Reeng} Автоматизированный реинжиниринг программ / Под ред. проф. А.Н. Терехова и А.А. Терехова. - СПб.: Издательство С.-Петербургского университета, 2000. 332~с.

        \bibitem {DrgBook} \emph {Ахо А., Сети Р., Ульман Дж.} Компиляторы: принципы, технологии, инструменты.  М:. Издательский дом <Вильямс>2003. 768~с.

        \bibitem {Martinenko} \emph {Мартыненко Б.К.} Языки и трансляции. — СПб.: Издательство С.-Петербургского университета, 2002. — 229~с.
        \bibitem {Diploma} \emph{Чемоданов И.С.} Генератор синтаксических анализаторов  для решения задач автоматизированного реинжиниринга программ. 2007. 37~c.        

        \bibitem {CCReview} \emph{Чемоданов И.С., Дубчук Н.П.} Обзор современных средств автоматизации создания синтаксических анализаторов // Системное программирование. - СПб.: Изд-во С.-Петерб. ун-та, 2006. 286-316~с.


        \bibitem {Practical Guide} \emph{Dick Grune, Ceriel Jacobs} PARSING TECHNIQUES A Practical Guide

        \bibitem {RECURSIVE-ASCENT PARSING} \emph {Larry Morell, David Middleton} RECURSIVE-ASCENT PARSING. Arkansas Tech University Russellville, Arkansas. 

        \bibitem {RecursiveAscentParsing} \emph {Lex Augusteijn} Recursive Ascent Parsing (Re: Parsing techniques). lex@prl.philips.nl (Lex Augusteijn) Mon, 10 May 1993 07:03:39 GMT 


        \bibitem {CurrentParsTechn} \emph{Mark G.J. van den Brand, Alex Sellink, Chris Verhoef} 
                Current Parsing Techniques in Software Renovation Considered Harmful.// IWPC '98: Proceedings of the 6th International Workshop on Program Comprehension. - IEEE Computer Society, Washington,1998.
        
        \bibitem {ISOEBNF} \emph ISO/IEC 14977 : 1996(E)

        \bibitem {Non-det-rec-asc} \emph {Rene Leermakers} Non-deterministic Recursive Ascent Parsing. Philips Research Laboratories, P.O. Box 80.000, 5600 JA Eindhoven, The Netherlands. 


        \bibitem {Jade} \emph {Ronald Veldena} Jade, a recursive ascent LALR(1) parser generator. September 8,1998




        \bibitem{APaGeD} http://apaged.mainia.de/ (сайт разработчиков APaGeD)

        \bibitem{DParser} http://dparser.sourceforge.net/ (сайт разработчиков DParser)

        \bibitem{Dypgen} http://dypgen.free.fr/  (сайт разработчиков Dypgen)

        \bibitem{h8me} http://parsing.codeplex.com/ (сайт разработчиков eu.h8me.Parsing)

        \bibitem{SmaCC} http://refactory.com/Software/SmaCC/ (сайт разработчиков SmaCC)

        \bibitem{DMS} http://semanticdesigns.com/Products/DMS/DMSToolkit.html (сайт разработчиков DMS)

        \bibitem{GDK} http://sourceforge.net/projects/gdk/ (сайт разработчиков GDK)

        \bibitem{Tom}       http://www-2.cs.cmu.edu/afs/cs/project/ai-repository/ai/areas/nlp/parsing/tom/0.html (сайт разработчиков Tom)

        \bibitem {Bison}    http://www.gnu.org/software/bison (сайт разработчиков Bison)

        \bibitem {Haskell}  http://www.haskell.org/ (дистрибутивы и документация по языку Haskell)

        \bibitem {Happy}    http://www.haskell.org/happy/ (сайт разработчиков Happy)

        \bibitem {ASF+SDF}  http://www.meta-environment.org (сайт разработчиков ASF+SDF)

        \bibitem {.NET}     http://www.microsoft.com/NET/ (сайт платформы .NET)  

        \bibitem{Wormhole}  http://www.mightyheave.com/blog/?p=270 (сайт разработчиков Wormhole)

        \bibitem {FS}       http://www.research.microsoft.com/fsharp (дистрибутивы и документация по языку F\#)             

        \bibitem {Elkhound} http://www.scottmcpeak/elkhound/ (сайт разработчиков Elkhound )

        \bibitem{UltraGram} http://www.ultragram.com/ (сайт разработчиков UltraGram)

\end{thebibliography}


\end{document}
