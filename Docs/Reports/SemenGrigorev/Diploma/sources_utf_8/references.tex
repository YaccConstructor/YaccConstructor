\begin{thebibliography}{50}

        \bibitem {Reeng} Автоматизированный реинжиниринг программ / Под ред. проф. А.Н. Терехова и А.А. Терехова. - СПб.: Издательство С.-Петербургского университета, 2000. 332~с.

        \bibitem {DrgBook} \emph {Ахо А., Сети Р., Ульман Дж.} Компиляторы: принципы, технологии, инструменты.  М:. Издательский дом <Вильямс>2003. 768~с.

        \bibitem {Martinenko} \emph {Мартыненко Б.К.} Языки и трансляции. — СПб.: Издательство С.-Петербургского университета, 2002. — 229~с.
        \bibitem {Diploma} \emph{Чемоданов И.С.} Генератор синтаксических анализаторов  для решения задач автоматизированного реинжиниринга программ. 2007. 37~c.        

        \bibitem {CCReview} \emph{Чемоданов И.С., Дубчук Н.П.} Обзор современных средств автоматизации создания синтаксических анализаторов // Системное программирование. - СПб.: Изд-во С.-Петерб. ун-та, 2006. 286-316~с.


        \bibitem {Practical Guide} \emph{Dick Grune, Ceriel Jacobs} PARSING TECHNIQUES A Practical Guide

        \bibitem {RECURSIVE-ASCENT PARSING} \emph {Larry Morell, David Middleton} RECURSIVE-ASCENT PARSING. Arkansas Tech University Russellville, Arkansas. 

        \bibitem {RecursiveAscentParsing} \emph {Lex Augusteijn} Recursive Ascent Parsing (Re: Parsing techniques). lex@prl.philips.nl (Lex Augusteijn) Mon, 10 May 1993 07:03:39 GMT 


        \bibitem {CurrentParsTechn} \emph{Mark G.J. van den Brand, Alex Sellink, Chris Verhoef} 
                Current Parsing Techniques in Software Renovation Considered Harmful.// IWPC '98: Proceedings of the 6th International Workshop on Program Comprehension. - IEEE Computer Society, Washington,1998.
        
        \bibitem {ISOEBNF} \emph ISO/IEC 14977 : 1996(E)

        \bibitem {Non-det-rec-asc} \emph {Rene Leermakers} Non-deterministic Recursive Ascent Parsing. Philips Research Laboratories, P.O. Box 80.000, 5600 JA Eindhoven, The Netherlands. 


        \bibitem {Jade} \emph {Ronald Veldena} Jade, a recursive ascent LALR(1) parser generator. September 8,1998




        \bibitem{APaGeD} http://apaged.mainia.de/ (сайт разработчиков APaGeD)

        \bibitem{DParser} http://dparser.sourceforge.net/ (сайт разработчиков DParser)

        \bibitem{Dypgen} http://dypgen.free.fr/  (сайт разработчиков Dypgen)

        \bibitem{h8me} http://parsing.codeplex.com/ (сайт разработчиков eu.h8me.Parsing)

        \bibitem{SmaCC} http://refactory.com/Software/SmaCC/ (сайт разработчиков SmaCC)

        \bibitem{DMS} http://semanticdesigns.com/Products/DMS/DMSToolkit.html (сайт разработчиков DMS)

        \bibitem{GDK} http://sourceforge.net/projects/gdk/ (сайт разработчиков GDK)

        \bibitem{Tom}       http://www-2.cs.cmu.edu/afs/cs/project/ai-repository/ai/areas/nlp/parsing/tom/0.html (сайт разработчиков Tom)

        \bibitem {Bison}    http://www.gnu.org/software/bison (сайт разработчиков Bison)

        \bibitem {Haskell}  http://www.haskell.org/ (дистрибутивы и документация по языку Haskell)

        \bibitem {Happy}    http://www.haskell.org/happy/ (сайт разработчиков Happy)

        \bibitem {ASF+SDF}  http://www.meta-environment.org (сайт разработчиков ASF+SDF)

        \bibitem {.NET}     http://www.microsoft.com/NET/ (сайт платформы .NET)  

        \bibitem{Wormhole}  http://www.mightyheave.com/blog/?p=270 (сайт разработчиков Wormhole)

        \bibitem {FS}       http://www.research.microsoft.com/fsharp (дистрибутивы и документация по языку F\#)             

        \bibitem {Elkhound} http://www.scottmcpeak/elkhound/ (сайт разработчиков Elkhound )

        \bibitem{UltraGram} http://www.ultragram.com/ (сайт разработчиков UltraGram)

\end{thebibliography}
