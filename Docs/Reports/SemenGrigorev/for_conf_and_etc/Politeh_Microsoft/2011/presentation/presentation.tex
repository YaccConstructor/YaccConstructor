\documentclass{beamer}
\usepackage{beamerthemesplit}
\usetheme{SPbGU}
%{CambridgeUS}
% Выпишем часть возможных стилей, некоторые из них могут содержать
% дополнительные опции
% Darmstadt, Ilmenau, CambridgeUS, default, Bergen, Madrid, AnnArbor,Pittsburg, Rochester,
% Antiles, Montpellier, Berkley, Berlin
\usepackage{pdfpages}
\usepackage{amsmath}
\usepackage{cmap} % for serchable pdf's
\usepackage[T2A]{fontenc} 
\usepackage[utf8]{inputenc}
\usepackage[english,russian]{babel}
\usepackage{indentfirst}
\usepackage{amsmath}
\usepackage{dot2texi}
\usepackage{tikz}
\usepackage{graphicx}

\usetikzlibrary{shapes,arrows}
% Если у вас есть логотип вашей кафедры, факультета или университета, то
% его можно включить в презентацию.

%\usefoottemplate{\vbox{}}%  \tinycolouredline{structure!25}% {\color{white}\textbf{\insertshortauthor\hfill% \insertshortinstitute}}% \tinycolouredline{structure}% {\color{white}\textbf{\insertshorttitle}\hfill}% }}

%\logo{\includegraphics[width=1cm]{SPbGU_Logo.png}}

%[GLR-анализатор]
\title[]{Генератор анализаторов с поддержкой неоднозначных атрибутных EBNF-грамматик в среде .NET}
\institute[СПбГУ]{
Санкт-Петербургский государственный университет \\
Математико-Механический факультет \\
Кафедра системного программирования }
%[Лукичёв А.С. Григорьев С.В.]


\author[Лукичёв А.С. Григорьев С.В.]{Григорьев Семён Вячеславович \\
  \and  
  {\bfseries Научный руководитель:} к.ф.-м.н. А.С. Лукичёв \\ 
}

\date{23 марта 2011г.}

\begin{document}
\sloppy
{

\begin{frame}
\begin{center}
{\includegraphics[width=1cm]{SPbGU_Logo.png}}
\end{center}
%\frame{ 
\titlepage
%}
\end{frame}
}

\begin{frame}
	\transwipe[direction=90]
	\frametitle{Область применения}
Реинжиниринг программного обеспечения:
	\begin{itemize}
		\item Упрощение создания и сопровождения грамматик
			\begin{itemize}
				\item Нет необходимости задавать однозначную контекстно-свободную
				\item Нет десятков конфликтов при одном изменении
			\end{itemize}
		\item Работа с диалектами одного языка
			\begin{itemize}
				\item Задание общей грамматики
				\item Автоматическое определение диалекта
			\end{itemize}		
	\end{itemize}
\end{frame}


\begin{frame}
	\transwipe[direction=90]
	\frametitle{Цели и задачи}
    \begin{block}{Цель}
        Разработка генератора синтаксических анализаторов для \bfseries{\underline {среды .NET}} \normalfont со следующими свойствами:  
	    \begin{itemize}
		    \item Работа с произвольными контекстно-свободными грамматиками
		    \item Поддержка EBNF-грамматик
		    \item Поддержка семантических вычислений
	    \end{itemize}
    \end{block}
    \begin{block}{Задача}
        Реализовать поддержку s-атрибутных и l-атрибутных грамматик.
    \end{block}
\end{frame}

\begin{frame}
	\transwipe[direction=90]
	\frametitle{Алгоритм}
	 GLR-анализатор предназначен для работы с произвольной {\bfseries{\underline {(в том числе неоднозначной!)}}} КС  грамматикой
	\begin{itemize}
		\item Для однозначных грамматик работает за линейное время
    \item {$O(n^{3})$ в худшем случае}
	\end{itemize}
	Рассмотренные подходы:
	\begin{itemize}
			\item Алгоритм Эрли				
			\item Алгоритм Томиты				
			\item Рекурсивно-восходящий алгоритм				
	\end{itemize}					
\end{frame}

\begin{frame}
	\transwipe[direction=90]
	\frametitle{EBNF-грамматики}
	Конструкции регулярных выражений в правых частях правил.
  \\
	Пример грамматики: 
		\begin{itemize}
			\item $S \rightarrow A(+A)*$
			\item $A \rightarrow a$
		\end{itemize}
	Преобразованная грамматика:
  \begin{itemize}
		\item $S \rightarrow AB$
		\item $A \rightarrow a$
		\item $B \rightarrow +AB$		
		\item $B \rightarrow \varepsilon$
	\end{itemize}
	Входная цепочка: a+a

\end{frame}

\begin{frame}[t]
	\transwipe[direction=90]
	\frametitle{EBNF-грамматики}
  \begin{center}
\begin{center}
	\begin{tabular}{ll}
		Ожидания пользователя: & Результат: \\
		\begin{dot2tex}[dot,autosize]
     digraph G
      {
         A1[label = "A"];
         p[label = "+"];
         A2[label = "A"];
         a1[label = "a"];
         a2[label = "a"];
         S -> A1 -> a1;
         S -> p;
         S -> A2 -> a2;
      }
      \end{dot2tex} 
    & 
          \begin{dot2tex}[dot,autosize]

      digraph G
      {
         A1[label = "A"];
         p[label = "+"];
         A2[label = "A"];
         a1[label = "a"];
         a2[label = "a"];         
         B1[label = "B"];
         B2[label = "B"];

         S -> A1 -> a1;
         S -> B1 -> p;
         B1 -> A2 ->a2;
         B1 -> B2 -> e;
       
      }
      \end{dot2tex} 
 \\
	\end{tabular}
	\label{tab:}
\end{center}
	%\begin{columns}
   %\begin{column}{5cm}
    % Ожидания пользователя:
     %\vspace{2cm}
		 %\begin{dot2tex}[dot,autosize]

%     digraph G
 %     {
  %       A1[label = "A"];
   %      p[label = "+"];
    %     A2[label = "A"];
     %    a1[label = "a"];
      %   a2[label = "a"];
       %  S -> A1 -> a1;
        % S -> p;
         %S -> A2 -> a2;
%      }
 %     \end{dot2tex}
  %  \end{column}
    %\pause
   % \begin{column}{5cm}
    % Результат: \\
    %
     % \begin{dot2tex}[dot,autosize]
%
 %     digraph G
  %    {
   %      A1[label = "A"];
    %     p[label = "+"];
     %    A2[label = "A"];
      %   a1[label = "a"];
       %  a2[label = "a"];         
        % B1[label = "B"];
         %B2[label = "B"];
%
 %        S -> A1 -> a1;
  %       S -> B1 -> p;
   %      B1 -> A2 ->a2;
    %     B1 -> B2 -> e;
     %  
      %}
      %\end{dot2tex} 
 % \\
 % \end{column}
%	\end{columns}
	%\label{tab:}
\end{center}



   
\end{frame}

\begin{frame}
	\transwipe[direction=90]
	\frametitle{Вычисление атрибутов}
    Правило грамматики:
  \begin{verbatim}someRule : val1 = (a {action1})* val2 = c {someFunc val1 val2};
  \end{verbatim}
\begin{center}
	\begin{tabular}{ll}
     Узел дерева вывода: & Дерево разбора $Str$: 
     \\
		 \begin{dot2tex}[dot,autosize]

      digraph string_of_child
      {
                S[label = "someRule"]

                c[label = "c"]; 
                a1[label = "a"];
                a2[label = "a"];
                a3[label = "a"];           
                  
                S -> c;                            
                S -> a1;
                S -> a2;
                S -> a3;

              subgraph cluster_STR
              {                                                
                      bgcolor = grey;
                      str[label = "",texlbl = "$Str:$",shape = plaintext]
                      c
                      a1;
                      a2;
                      a3;
                      
              };
      }
      \end{dot2tex}
    &        
      \begin{dot2tex}[dot,autosize]

      digraph string_diriv_tree
      {

                Seq[label = "Seq"]
                Cls[label = "Cls"]
                a1[label = "a"];
                a2[label = "a"];
                a3[label = "a"];
                c[label = "c"];                        
                                       
                Seq -> Cls;            
                Seq -> c; 
                Cls -> a1;
                Cls -> a2; 
                Cls -> a3;                           

      }
      \end{dot2tex} 
  \\
	\end{tabular}
	\label{tab:}
\end{center}



  
\end{frame}

\begin{frame}
	\transwipe[direction=90]
	\frametitle{Вычисление атрибутов}
  \begin{itemize}
   \item
    Конечный автомат с помеченными переходами:\\
    %\begin{center}
    \begin{dot2tex}[dot]
    digraph G
    {
            rankdir = LR
            S -> F [  label="123a/123l"
                    , texlbl = "$someSymbol/someLable$" ]
    }
    \end{dot2tex}
%	\end{center}

   \item
    Трасса автомата: \\
    $[(SeqS,1);$\\             
    $\phantom \qquad(ClsS,1);$\\
    $\phantom \qquad \qquad \qquad (SeqS,2); (LeafS,4); \ 'a'; (LeafE,4); ... (SeqE,2);$ \\ 
    $\phantom \qquad (ClsE,1); $\\
    $\phantom \qquad (LeafS,4); \ 'c'; (LeafE,4);$\\
    $(SeqE,1)]$
   \item
    По трассе строится дерево разбора.
  \end{itemize}
\end{frame}

\begin{frame}
	\transwipe[direction=90]
	\frametitle{Результаты}
	Для генератора GLR-анализаторов с поддержкой EBNF-грамматик без преобразования реализован механизм поддержки l-атрибутных и s-атрибутных грамматик:
	\begin{itemize}
		\item генератор action-кода;
        \item механизм вычисления атрибутов;
	\end{itemize}	
\end{frame}


\begin{frame}
	\transwipe[direction=90]
	\frametitle{Используемые продукты Microsoft}
	 При разработке используются следующие продукты Microsoft:
	\begin{itemize}
		\item Microsoft Visual Studio 2010 -- среда разработки
		\item F\# -- язык разработки и целевой язык генератора
        \item F\# PowerPack -- FsLex и FsYacc.
	\end{itemize}	
\end{frame}

\begin{frame}
	\transwipe[direction=90]
	\frametitle{Заключение}
        Данная разаработка ведётся в рамках проекта кафедры системного программарования Математико-Механического факультета СПбГУ YaccConstructor. Исходный код и дополнительную информацию по проекту можно найти на сайте \href{http://code.google.com/p/recursive-ascent/}{http://code.google.com/p/recursive-ascent}.

\end{frame}

\end{document}
